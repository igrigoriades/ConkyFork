\documentclass[10pt,a4paper]{report}
\usepackage[utf8]{inputenc}
\usepackage{amsmath}
\usepackage{amsfonts}
\usepackage{amssymb}
\usepackage{xcolor}
\usepackage{hyperref}

\author{GG}
\title{README}
\date{}
\begin{document}
\maketitle

\section*{SysMonV1.0}	%Using * we get rid of the numbering before the text
SysMon is a script written in bash. It's purpose is to fork conky into the terminal for headless usage. 
SysMon consists of four files including:
\begin{itemize}
	\item{\textbf{conkyrc:}} Conkyrc is the configuration file of the conky itself. Be sure to use out\_to\_x=false and
	out\_to\_console=true in conky.config section. You can create your own rc file or modify the existing.

	\item{\textbf{sysmon-deamon:}} Sysmon-deamon is a script written in bash that using the watch command   
	refreshes the output of the sysmon-instance, therefore the sysmon-instance runs repeatedly in background. 
	
	\item{\textbf{sysmon-instance:}} Sysmon-instance is a script written in bash that runs the conky, while writing 
	its output to a file. Afterwords it goes through the file writing to the terminal its contents using
	LS\_COLORS  color scheme 

	\item{\textbf{tmp:} } Tmp is the file that the conky writes its output. It is overwritten each time.
\end{itemize}

\section*{Usage} Make the sysmon-deamon and sysmon-instance executable using "chmod +x" command
and run sysmon-deamon.

\section*{License} Conky is licensed under the terms of the GPLv3 license therefore sysmon is also licensed under the
terms of GPLv3. A copy of the license is located under the "Documentation" directory or can be found on the 
following{\textcolor{blue}{URL}}.

\end{document}
